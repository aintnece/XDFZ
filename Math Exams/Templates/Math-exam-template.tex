\documentclass{sbs-exam}

\usepackage{ulem}
\usepackage{array}
\usepackage{enumitem}
\usepackage{tabularx}
\usepackage{ulem}
\usepackage{xcolor}
\usepackage{multirow}
\usepackage{graphicx} % 图形支持包(必须)
\usepackage{fancyhdr} % 页眉控制包(必须)
\usepackage{titling} % 提供高级标题控制
\usepackage{lmodern} % 现代字体支持
\usepackage{amsmath, amssymb} % 数学符号和公式支持
\usepackage{lipsum} % 用于生成示例文本

% ===== 资源路径配置 =====
\graphicspath{
  {../Assets/}     % 公共资源
  {../Media/}      % 试卷专用资源
}

% ===== 考试元数据配置 =====
\examtitle{Monthly Exam, 2024-2025 T2}
\examsubject{Further Math}{May 2025}{60 Minutes}{Shi Feng}
\MarksBreakdown{5pts}{10}{5pts}{4}{30pts}


\begin{document}

\maketitle

\examnotice

\makemarksheet

\examtools

\newpage

\makeatletter
\section*{Part I: Multiple Choice Questions ($ \@MCQ \times \@MCQNo $)}
\makeatother
\begin{enumerate}
\item
If a root of $ f(x) = 0 $ lies in $[3, 4]$, what is the midpoint after the first iteration of interval bisection?\\
A) 3.25\\
B) 3.5\\
C) 3.75\\
D) 3.0

\item 
The formula for the first approximation $ x_1 $ using linear interpolation is:\\
A) $ x_1 = a - \frac{f(a)(b-a)}{f(b)-f(a)} $\\
B) $ x_1 = \frac{a + b}{2} $\\
C) $ x_1 = a - \frac{f(a)}{f'(a)} $\\
D) $ x_1 = \frac{f(b) - f(a)}{b - a} $

\end{enumerate}

\makeatletter
\section*{Part II: Short Answer Questions ($ \text{\@SAQ} \times \@SAQNo $)}
\makeatother

\begin{enumerate}
    
\item After 3 iterations of interval bisection on $[3, 4]$, the interval length is \underline{\hspace{2cm}}.

\end{enumerate}

\makeatletter
\section*{Part III: Long Answer Questions ($ \@LAQTotal $)}
\makeatother

\begin{enumerate}

\item 
Let $ z^6 = 1$.  
\begin{enumerate}
\item[(a)] Find all the solutions to the equation.

\vspace{8cm} % 答题空间

\item[(b)] Show each solution on an Argand diagram.

\vspace{4cm} % 答题空间

\item[(c)] Show that each solution lies on a circle with center $(0, 0)$ and radius $1$.

\vspace{4cm} % 答题空间

\end{enumerate}

\end{enumerate}



\end{document}