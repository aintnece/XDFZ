\documentclass{../Templates/sbs-exam}

\usepackage{ulem}
\usepackage{array}
\usepackage{enumitem}
\usepackage{tabularx}
\usepackage{ulem}
\usepackage{xcolor}
\usepackage{multirow}
\usepackage{multicol}
\usepackage{graphicx} % 图形支持包(必须)
\usepackage{fancyhdr} % 页眉控制包(必须)
\usepackage{titling} % 提供高级标题控制
\usepackage{lmodern} % 现代字体支持
\usepackage{amsmath, amssymb} % 数学符号和公式支持
\usepackage{lipsum} % 用于生成示例文本
\usepackage{caption} % 图表标题支持

% ===== 资源路径配置 =====
\graphicspath{
  {../Assets/}     % 公共资源
  {../Media/}      % 试卷专用资源
  {./midterm-g10-c1-fall-2025/} % 本试卷资源
}

% ===== 考试元数据配置 =====
\examtitle{Midterm Exam, 2025-2026 T1}
\examsubject{G10 C1 Math}{Nov 2025}{90 Minutes}{Shi Feng}
\MarksBreakdown{}{}{}{}{73 \text{ Marks}}{73 \text{ Marks}}

% 自定义命令用于处理记分内容
\newcommand{\imarks}[1]{\hfill \textbf{(#1 marks)}}
\newcommand{\totalmarks}[1]{\par\noindent\hfill \textbf{(Total for question = #1 marks)}\par}

\begin{document}

\maketitle

\examnotice

\makemarksheet

\examtools

\newpage

\makeatletter

\section*{Questions}

\begin{enumerate}
\item % Q1
\begin{enumerate}
\item Given that $3^{-1.5}=a\sqrt{3}$ find the exact value of $a$. \imarks{2}
\item Simplify fully $\frac{(2x^{2})^{3}}{4x^{2}}$. \imarks{2}
\end{enumerate}
\totalmarks{5}

\vspace{3cm}

\item % Q2
In this question you must show all stages of your working. Solutions relying entirely on calculator technology are not acceptable. \\
The curve C has equation
$$y=\frac{2}{x}-k$$
where $k$ is a positive constant.
\begin{enumerate}
\item Sketch the graph of C. \\
Show on your sketch \\
- the coordinates of any points of intersection of C with the coordinate axes \\
- the equation of the horizontal asymptote to C \\
stating each in terms of $k$. \imarks{3} \\
The line $l$ has equation $y=-kx-6$
\item Given that $l$ intersects C at 2 distinct points, find the range of possible values of $k$. \imarks{5}
\end{enumerate}
\totalmarks{8}

\vspace{3cm}

\item % Q3
The equation
$$\frac{3}{x}+5=-2x+c$$
where $c$ is a constant, has no real roots. \\
Find the range of possible values of $c$. \imarks{7}

\vspace{3cm}

\item % Q4
\begin{figure}[h]
\centering
\includegraphics[width=0.6\textwidth]{figure 1.png}
\caption*{Figure 1}
\end{figure}
Figure 1 shows a line $l_{1}$ with equation $2y=x$ and a curve C with equation
$$y=2x-\frac{1}{8}x^{2}$$
The region R, shown unshaded in Figure 1, is bounded by the line $l_{1}$, the curve C and a line $l_{2}$. Given that $l_{2}$ is parallel to the y-axis and passes through the intercept of C with the positive x-axis, identify the inequalities that define R. \imarks{3}

\vspace{5cm}


\item % Q5
\begin{enumerate}
\item On the same axes, sketch the graphs of $y=x+2$ and $y=x^{2}-x-6$ showing the coordinates of all points at which each graph crosses the coordinate axes. \imarks{4}
\item On your sketch, show, by shading, the region R defined by the inequalities
$$y<x+2 \text{ and } y>x^{2}-x-6$$ \imarks{1}
\item Hence, or otherwise, find the set of values of $x$ for which $x^{2}-2x-8<0$ \imarks{3}
\end{enumerate}
\totalmarks{8}

\vspace{5cm}

\item % Q6

Solve the simultaneous equations
$$\begin{array}{l} y+4x+1=0 \\ y^{2}+5x^{2}+2x=0 \end{array}$$ \imarks{6}

\vspace{5cm}

\item % Q7

A curve has equation
$$y=\frac{x^{3}}{6}+4\sqrt{x}-15 \quad x\geqslant 0$$
\begin{enumerate}
\item Find $\frac{dy}{dx}$, giving the answer in simplest form. \imarks{3}
\item The point $P\left(4,\frac{11}{3}\right)$ lies on the curve. Find the equation of the normal to the curve at P. Write your answer in the form $ax+by+c=0$, where $a$, $b$ and $c$ are integers to be found. \imarks{4}
\end{enumerate}
\totalmarks{7}

\vspace{5cm}

\item % Q8

\begin{figure}[h]
\centering
\includegraphics[width=0.5\textwidth]{figure 2.png}
\caption*{Figure 2}
\end{figure}
Figure 2 shows a plot of the curve with equation $y=\sin\theta$, $0\leq\theta\leq 360^{\circ}$
\begin{enumerate}
\item State the coordinates of the minimum point on the curve with equation
$$y=4\sin\theta, \quad 0\leq\theta\leq 360^{\circ}$$ \imarks{2}
\item A copy of Figure 3, called Diagram 1, is shown here. \\
\begin{figure}[h]
\centering
\includegraphics[width=0.5\textwidth]{figure 3.png}
\caption*{Diagram 1}
\end{figure}
On Diagram 1, sketch and label the curves
\begin{enumerate}
\item $y=1+\sin\theta, \quad 0\leq\theta\leq 360^{\circ}$ \imarks{1}
\item $y=\tan\theta, \quad 0\leq\theta\leq 360^{\circ}$ \imarks{1}
\end{enumerate}
\item Hence find the number of solutions of the equation
\begin{enumerate}
\item $\tan\theta=1+\sin\theta$ that lie in the region $0\leq\theta\leq 2160^{\circ}$ \imarks{1.5}
\item $\tan\theta=1+\sin\theta$ that lie in the region $0\leq\theta\leq 1980^{\circ}$ \imarks{1.5}
\end{enumerate}
\end{enumerate}
\totalmarks{7}

\vspace{4cm}

\item % Q9

\begin{figure}[h]
\centering
\includegraphics[width=0.6\textwidth]{figure 4.png}
\caption*{Figure 3}
\end{figure}
The shape ABCDA consists of a sector ABCOA of a circle, centre O, joined to a triangle AOD, as shown in Figure 3. \\
The point D lies on OC. \\
The radius of the circle is $6~cm$, length AD is $5~cm$ and angle AOD is 0.7 radians.
\begin{enumerate}
\item Find the area of the sector ABCOA, giving your answer to one decimal place. \imarks{3}
\item Given angle ADO is obtuse, find the size of angle ADO, giving your answer to 3 decimal places. \imarks{3}
\item Hence find the perimeter of shape ABCDA, giving your answer to one decimal place. \imarks{4}
\end{enumerate}
\totalmarks{10}

\vspace{4cm}

\item % Q10
\begin{enumerate}
\item On Diagram 2 sketch the graphs of
\begin{enumerate}
\item $y=x(3-x)$ \imarks{2}
\item $y=x(x-2)(5-x)$ \imarks{2}
\end{enumerate}
\begin{figure}[h]
\centering
\includegraphics[width=0.6\textwidth]{figure 5.png}
\caption*{Diagram 2}
\end{figure}
showing clearly the coordinates of the points where the curves cross the coordinate axes.
\item Show that the $x$ coordinates of the points of intersection of
$$y=x(3-x) \text{ and } y=x(x-2)(5-x)$$
are given by the solutions to the equation $x(x^{2}-8x+13)=0$ \imarks{3}
\item The point P lies on both curves. Given that P lies in the first quadrant, find, using algebra and showing your working, the exact coordinates of P. \imarks{5}
\end{enumerate}
\totalmarks{12}

\end{enumerate}

\end{document}