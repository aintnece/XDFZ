\documentclass{../Templates/sbs-exam}

\usepackage{ulem}
\usepackage{array}
\usepackage{enumitem}
\usepackage{tabularx}
\usepackage{ulem}
\usepackage{xcolor}
\usepackage{multirow}
\usepackage{multicol}
\usepackage{graphicx} % 图形支持包(必须)
\usepackage{fancyhdr} % 页眉控制包(必须)
\usepackage{titling} % 提供高级标题控制
\usepackage{lmodern} % 现代字体支持
\usepackage{amsmath, amssymb} % 数学符号和公式支持
\usepackage{lipsum} % 用于生成示例文本

% ===== 资源路径配置 =====
\graphicspath{
  {../Assets/}     % 公共资源
  {../Media/}      % 试卷专用资源
}

% ===== 考试元数据配置 =====
\examtitle{Monthly Exam, 2025-2026 T1}
\examsubject{Pure Mathematics 1}{Oct 2025}{90 Minutes}{Shi Feng}
\MarksBreakdown{2 \text{ Marks}}{15}{2 \text{ Marks}}{15}{40 \text{ Marks}}


\begin{document}

\maketitle

\examnotice

\makemarksheet

\examtools

\newpage

\makeatletter

\section*{Part I: Multiple Choice Questions (30 marks)}
Choose the correct answer. Each question carries 2 marks.

\begin{enumerate}
    \item Simplify \((3x^2 y^{-3})^2\):
    \begin{multicols}{4}
    \begin{enumerate}[label=(\alph*)]
        \item \(6x^4 y^{-6}\)
        \item \(9x^4 y^{-6}\)
        \item \(9x^4 y^{-5}\)
        \item \(6x^4 y^{-5}\)
    \end{enumerate}
    \end{multicols}

    \item Find the value of \(k\) if \(x^2 - 4x + k = 0\) has discriminant 0:
    \begin{multicols}{4}
    \begin{enumerate}[label=(\alph*)]
        \item 2
        \item 4
        \item 6
        \item 8
    \end{enumerate}
    \end{multicols}

    \item Solve \(x^2 - 5x - 14 > 0\):
    \begin{multicols}{4}
    \begin{enumerate}[label=(\alph*)]
        \item \(x < -2\) or \(x > 7\)
        \item \(x < -7\) or \(x > 2\)
        \item \(-2 < x < 7\)
        \item \(-7 < x < 2\)
    \end{enumerate}
    \end{multicols}

    \item The graph of \(y = \frac{1}{x}\) is translated by \(\begin{pmatrix} 2 \\ -3 \end{pmatrix}\). What is the new equation?
    \begin{multicols}{4}
    \begin{enumerate}[label=(\alph*)]
        \item \(y = \frac{1}{x-2} - 3\)
        \item \(y = \frac{1}{x+2} + 3\)
        \item \(y = \frac{1}{x-2} + 3\)
        \item \(y = \frac{1}{x+2} - 3\)
    \end{enumerate}
    \end{multicols}

    \item Which is a factor of \(x^3 - 4x\)?
    \begin{multicols}{4}
    \begin{enumerate}[label=(\alph*)]
        \item \(x - 2\)
        \item \(x + 1\)
        \item \(x^2 + 4\)
        \item \(x - 4\)
    \end{enumerate}
    \end{multicols}

    \item Simplify \(\sqrt{12} + \sqrt{27}\):
    \begin{multicols}{4}
    \begin{enumerate}[label=(\alph*)]
        \item \(5\sqrt{3}\)
        \item \(6\sqrt{3}\)
        \item \(7\sqrt{3}\)
        \item \(8\sqrt{3}\)
    \end{enumerate}
    \end{multicols}

    \item Evaluate \(8^{-\frac{2}{3}}\):
    \begin{multicols}{4}
    \begin{enumerate}[label=(\alph*)]
        \item \(\frac{1}{4}\)
        \item \(\frac{1}{2}\)
        \item 2
        \item 4
    \end{enumerate}
    \end{multicols}

    \item The function \(f(x) = x^2 + 2x + 3\) has:
    \begin{multicols}{4}
    \begin{enumerate}[label=(\alph*)]
        \item Two distinct real roots
        \item One repeated real root
        \item No real roots
        \item A maximum point at \((-1, 2)\)
    \end{enumerate}
    \end{multicols}

    \item Complete the square: \(x^2 + 8x + 1\)
    \begin{multicols}{4}
    \begin{enumerate}[label=(\alph*)]
        \item \((x+4)^2 - 15\)
        \item \((x+4)^2 + 15\)
        \item \((x+8)^2 - 63\)
        \item \((x+8)^2 + 1\)
    \end{enumerate}
    \end{multicols}

    \item Solve \(x^2 - 6x + 9 = 0\):
    \begin{multicols}{4}
    \begin{enumerate}[label=(\alph*)]
        \item \(x = 3\) only
        \item \(x = -3\) only
        \item \(x = 3\) and \(x = -3\)
        \item No real solutions
    \end{enumerate}
    \end{multicols}

    \item Simplify \(\frac{\sqrt{50} + \sqrt{18}}{\sqrt{2}}\):
    \begin{multicols}{4}
    \begin{enumerate}[label=(\alph*)]
        \item \(4\)
        \item \(4\sqrt{2}\)
        \item \(8\)
        \item \(2\sqrt{2}\)
    \end{enumerate}
    \end{multicols}

    \item Evaluate \(27^{-\frac{2}{3}}\):
    \begin{multicols}{4}
    \begin{enumerate}[label=(\alph*)]
        \item \(9\)
        \item \(\frac{1}{9}\)
        \item \(\frac{1}{3}\)
        \item \(3\)
    \end{enumerate}
    \end{multicols}

    \item Rationalize: \(\frac{-3}{2+\sqrt{5}}\)
    \begin{multicols}{4}
    \begin{enumerate}[label=(\alph*)]
        \item \(6+3\sqrt{5}\)
        \item \(2-\sqrt{5}\)
        \item \(6-3\sqrt{5}\)
        \item \(\frac{3\sqrt{5}-6}{3}\)
    \end{enumerate}
    \end{multicols}

    \item Complete the square: \(x^2 + 6x + 5\)
    \begin{multicols}{4}
    \begin{enumerate}[label=(\alph*)]
        \item \((x+3)^2 - 4\)
        \item \((x+3)^2 + 4\)
        \item \((x+6)^2 - 31\)
        \item \((x+6)^2 + 5\)
    \end{enumerate}
    \end{multicols}

    \item The graph of \(y = x^2\) is translated by \(\begin{pmatrix} -2 \\ 3 \end{pmatrix}\). What is the new equation?
    \begin{multicols}{4}
    \begin{enumerate}[label=(\alph*)]
        \item \(y = (x-2)^2 + 3\)
        \item \(y = (x+2)^2 + 3\)
        \item \(y = (x-2)^2 - 3\)
        \item \(y = (x+2)^2 - 3\)
    \end{enumerate}
    \end{multicols}
\end{enumerate}

\section*{Part II: Short Answer Questions (30 marks)}
\subsection*{True/False Questions (10 marks)}
State whether each statement is True or False. Each question carries 2 marks.

\begin{enumerate}
    \item \textbf{True/False:} The function \(f(x) = x^2 + 2x + 3\) has no real roots.
    \item \textbf{True/False:} The graph of \(y = (x-2)^2 + 3\) has a minimum point at \((2, 3)\).
    \item \textbf{True/False:} \(\sqrt{12} + \sqrt{27} = 5\sqrt{3}\).
    \item \textbf{True/False:} The inequality \(\frac{1}{x} > 2\) holds for all \(x > 0\).
    \item \textbf{True/False:} The equations \(y = x^2\) and \(y = 2x - 1\) have exactly one solution.
\end{enumerate}

\subsection*{Fill in the Blanks (20 marks)}
Complete each statement. Each blank carries 2 marks.

\begin{enumerate}
    \item Simplify \(\frac{6x^3 y^2}{2xy}\): \rule{3cm}{0.15mm}
    \item Solve \(x^2 - 6x + 9 = 0\): \rule{3cm}{0.15mm}
    \item Complete the square: \(x^2 + 8x + 1 =\) \rule{3cm}{0.15mm}
    \item Solve \(\frac{x+1}{x-2} < 0\): \rule{3cm}{0.15mm}
    \item Asymptotes of \(y = \frac{3}{x-4} + 2\): \rule{3cm}{0.15mm} and \rule{3cm}{0.15mm}
    \item Discriminant of \(3x^2 - 2x + 5 = 0\): \rule{3cm}{0.15mm}
    \item Roots of \(2x^2 + 5x - 3 = 0\): \rule{3cm}{0.15mm} and \rule{3cm}{0.15mm}
    \item The inequality \(x^2 + 4x + 4 \geq 0\) holds for \rule{3cm}{0.15mm}
    \item Translation vector mapping \(y = x^2\) to \(y = (x+3)^2 - 5\): \rule{3cm}{0.15mm}
    \item For \(f(x) = (2x-3)(x+1)\), find \(f(2x)\): \rule{3cm}{0.15mm}
\end{enumerate}

\section*{Part III: Long Answer Questions (40 marks)}
Show all working. Each question carries 10 marks.

\begin{enumerate}
    \item \textbf{Quadratic Equations and Inequalities}
    \begin{enumerate}
        \item Solve \(x^2 - 10x + 16 = 0\) by factorisation. (3 marks)
        \item Hence, solve \(x^2 - 10x + 16 \leq 0\). (3 marks)
        \item Sketch the graph of \(y = x^2 - 10x + 16\), indicating key points. (4 marks)
    \end{enumerate}
    \vspace{8cm}

    \item \textbf{Algebraic Manipulation}
    \begin{enumerate}
        \item Simplify \(\frac{2x^2 - 8}{x^2 - 4x + 4}\). (3 marks)
        \item Express \(\frac{3}{\sqrt{5} - 1}\) in the form \(a + b\sqrt{5}\), where \(a, b\) are rational. (3 marks)
        \item Factorise completely: \(x^3 - 3x^2 - 4x + 12\). (4 marks)
    \end{enumerate}
    \vspace{8cm}

    \item \textbf{Graphs and Transformations}
    \begin{enumerate}
        \item Sketch \(y = x^2 - 4x + 3\), labeling intercepts and turning point. (4 marks)
        \item On the same axes, sketch \(y = x(x-1)(x+1)\), labeling intercepts. (3 marks)
        \item Find the translation vector mapping \(y = x^2\) to \(y = (x-1)^2 + 2\). (3 marks)
    \end{enumerate}
    \vspace{8cm}

    \item \textbf{Advanced Problems}
    \begin{enumerate}
        \item Sketch these functions, labeling intercepts: (4 marks)
        \[f(x) = (2+x)^2(2-x), \quad g(x) = \frac{-4}{x^2}\]
        \item State the number of solutions to \(f(x) = g(x)\) and explain why. (3 marks)
        \item Shade the region satisfying \(y \geq f(x)\) and \(y \leq g(x)\). (3 marks)
    \end{enumerate}
\end{enumerate}


\end{document}