\documentclass{../Templates/sbs-exam}

\usepackage{ulem}
\usepackage{array}
\usepackage{enumitem}
\usepackage{tabularx}
\usepackage{ulem}
\usepackage{xcolor}
\usepackage{multirow}
\usepackage{multicol}
\usepackage{graphicx} % 图形支持包(必须)
\usepackage{fancyhdr} % 页眉控制包(必须)
\usepackage{titling} % 提供高级标题控制
\usepackage{lmodern} % 现代字体支持
\usepackage{amsmath, amssymb} % 数学符号和公式支持
\usepackage{lipsum} % 用于生成示例文本

% ===== 资源路径配置 =====
\graphicspath{
  {../Assets/}     % 公共资源
  {../Media/}      % 试卷专用资源
}

% ===== 考试元数据配置 =====
\examtitle{Monthly Exam, 2025-2026 T1}
\examsubject{Pure Mathematics 1}{Oct 2025}{90 Minutes}{Shi Feng}
\MarksBreakdown{2 \text{ Marks}}{15}{2 \text{ Marks}}{15}{40 \text{ Marks}}


\begin{document}

\maketitle

\examnotice

\makemarksheet

\examtools

\newpage

\makeatletter

\section*{Part I: Multiple Choice Questions (30 marks)}
Choose the correct answer. Each question carries 2 marks.

\begin{enumerate}
    \item Simplify the expression: $(2x^2)^3$
    \begin{multicols}{4}
    \begin{enumerate}[label=(\alph*)]
        \item $6x^5$
        \item $8x^5$
        \item $6x^6$
        \item $8x^6$
    \end{enumerate}
    \end{multicols}

    \item What is the value of $8^{\frac{2}{3}}$?
    \begin{multicols}{4}
    \begin{enumerate}[label=(\alph*)]
        \item 2
        \item 4
        \item 16
        \item 64
    \end{enumerate}
    \end{multicols}

    \item The simplified form of $\sqrt{48}$ is:
    \begin{multicols}{4}
    \begin{enumerate}[label=(\alph*)]
        \item $4\sqrt{3}$
        \item $6\sqrt{2}$
        \item $8\sqrt{3}$
        \item $12\sqrt{2}$
    \end{enumerate}
    \end{multicols}

    \item Which is a factor of $x^2 - 5x + 6$?
    \begin{multicols}{4}
    \begin{enumerate}[label=(\alph*)]
        \item $x + 2$
        \item $x - 3$
        \item $x + 6$
        \item $x - 1$
    \end{enumerate}
    \end{multicols}

    \item The solution to $2x - 5 = 11$ is:
    \begin{multicols}{4}
    \begin{enumerate}[label=(\alph*)]
        \item $x = 3$
        \item $x = 6$
        \item $x = 8$
        \item $x = 13$
    \end{enumerate}
    \end{multicols}

    \item The graph of $y = x^2 - 4$ has its turning point at:
    \begin{multicols}{4}
    \begin{enumerate}[label=(\alph*)]
        \item $(0, -4)$
        \item $(0, 4)$
        \item $(2, 0)$
        \item $(-2, 0)$
    \end{enumerate}
    \end{multicols}

    \item What is the discriminant of $x^2 + 4x + 4 = 0$?
    \begin{multicols}{4}
    \begin{enumerate}[label=(\alph*)]
        \item 0
        \item 4
        \item 8
        \item 16
    \end{enumerate}
    \end{multicols}

    \item The simplified form of $\frac{3}{\sqrt{2}}$ is:
    \begin{multicols}{4}
    \begin{enumerate}[label=(\alph*)]
        \item $\frac{3\sqrt{2}}{2}$
        \item $\frac{\sqrt{6}}{2}$
        \item $\frac{3}{2}\sqrt{2}$
        \item $\sqrt{6}$
    \end{enumerate}
    \end{multicols}

    \item Which represents a translation of $y = x^2$ by \(\begin{pmatrix} 0 \\ 3 \end{pmatrix}\)?
    \begin{multicols}{4}
    \begin{enumerate}[label=(\alph*)]
        \item $y = (x+3)^2$
        \item $y = x^2 + 3$
        \item $y = (x-3)^2$
        \item $y = x^2 - 3$
    \end{enumerate}
    \end{multicols}

    \item The solution set for $x^2 - 9 < 0$ is:
    \begin{multicols}{4}
    \begin{enumerate}[label=(\alph*)]
        \item $x < -3$
        \item $x > 3$
        \item $-3 < x < 3$
        \item $x < -3$ or $x > 3$
    \end{enumerate}
    \end{multicols}

    \item The function $f(x) = x^2 + 2x + 1$ has:
    \begin{multicols}{4}
    \begin{enumerate}[label=(\alph*)]
        \item Two distinct real roots
        \item One repeated real root
        \item No real roots
        \item A maximum point
    \end{enumerate}
    \end{multicols}

    \item Simplify: $(3x^2y)(4xy^3)$
    \begin{multicols}{4}
    \begin{enumerate}[label=(\alph*)]
        \item $7x^3y^4$
        \item $12x^2y^3$
        \item $12x^3y^4$
        \item $7x^2y^3$
    \end{enumerate}
    \end{multicols}

    \item The value of $16^{-\frac{1}{2}}$ is:
    \begin{multicols}{4}
    \begin{enumerate}[label=(\alph*)]
        \item -4
        \item $\frac{1}{4}$
        \item 4
        \item $-\frac{1}{4}$
    \end{enumerate}
    \end{multicols}

    \item Which is equivalent to $\sqrt{20} + \sqrt{45}$?
    \begin{multicols}{4}
    \begin{enumerate}[label=(\alph*)]
        \item $5\sqrt{5}$
        \item $13\sqrt{5}$
        \item $5\sqrt{13}$
        \item $6\sqrt{5}$
    \end{enumerate}
    \end{multicols}
\end{enumerate}

\section*{Part II: Short Answer Questions (40 marks)}

\subsection*{True or False (20 marks)}
State whether each statement is True or False. Each question carries 2 marks.

\begin{enumerate}
    \item \textbf{True/False:} $\sqrt{9 + 16} = \sqrt{9} + \sqrt{16}$
    \item \textbf{True/False:} $(x + 2)^2 = x^2 + 4$
    \item \textbf{True/False:} The expression $\frac{x^2 - 4}{x - 2}$ simplifies to $x + 2$ for all $x \neq 2$
    \item \textbf{True/False:} $2^{-3} = -8$
    \item \textbf{True/False:} The quadratic formula is $x = \frac{-b \pm \sqrt{b^2 - 4ac}}{2a}$
    \item \textbf{True/False:} $y = (x-3)^2$ has its minimum point at $(3, 0)$
    \item \textbf{True/False:} $\frac{1}{x^{-2}} = x^2$
    \item \textbf{True/False:} The inequality $x^2 + 1 < 0$ has real solutions
    \item \textbf{True/False:} $27^{\frac{2}{3}} = 9$
    \item \textbf{True/False:} The graph of $y = \frac{1}{x}$ has asymptotes at $x=0$ and $y=0$
\end{enumerate}

\subsection*{Fill in the Blanks (20 marks)}
Complete each statement. Each blank carries 2 marks.

\begin{enumerate}
    \item The solutions to $x^2 - 5x + 6 = 0$ are \rule{2cm}{0.15mm} and \rule{2cm}{0.15mm}
    \item Complete the square: $x^2 + 6x + 1 = (x + \rule{1cm}{0.15mm})^2 - \rule{1cm}{0.15mm}$
    \item Simplify $\frac{6x^3y^2}{2xy} =$ \rule{3cm}{0.15mm}
    \item The discriminant of $2x^2 - 3x + 1 = 0$ is \rule{3cm}{0.15mm}
    \item Rationalize: $\frac{5}{\sqrt{3}} =$ \rule{3cm}{0.15mm}
    \item The turning point of $y = x^2 - 4x + 3$ is at (\rule{1cm}{0.15mm}, \rule{1cm}{0.15mm})
    \item $8^{\frac{1}{3}} =$ \rule{2cm}{0.15mm}
    \item The factors of $x^2 - 9$ are \rule{2cm}{0.15mm} and \rule{2cm}{0.15mm}
    \item The range of $y = x^2 + 2$ is \rule{3cm}{0.15mm}
    \item The translation that maps $y = x^2$ to $y = (x-2)^2 + 3$ is by vector \rule{2cm}{0.15mm}
\end{enumerate}

\section*{Part III: Long Answer Questions (30 marks)}
Show your working for each question.

\begin{enumerate}
    \item 
    \begin{enumerate}
        \item Simplify: $\frac{2x^2 - 8}{x^2 - 4x + 4}$ (3 marks)
        \item Expand and simplify: $(2x - 3)(x + 4) - (x - 2)^2$ (3 marks)
    \end{enumerate}
    \vspace{7cm}

    \item 
    \begin{enumerate}
        \item Solve by factorization: $x^2 - 7x + 12 = 0$ (3 marks)
        \item Solve using the quadratic formula: $2x^2 - 5x - 3 = 0$ (3 marks)
    \end{enumerate}
    \vspace{7cm}

    \item 
    \begin{enumerate}
        \item Sketch the graph of $y = x^2 - 4$, showing intercepts with axes (3 marks)
        \item Find the coordinates of the turning point of $y = x^2 + 6x + 1$ (3 marks)
    \end{enumerate}
    \vspace{5cm}

    \item 
    \begin{enumerate}
        \item Simplify: $\sqrt{50} + \sqrt{18} - \sqrt{8}$ (3 marks)
        \item Evaluate: $16^{\frac{3}{4}} + 8^{-\frac{2}{3}}$ (3 marks)
    \end{enumerate}
    \vspace{5cm}

    \item Sketch the graph of the following equations, clearly showing X and Y intercepts. (6 marks)
    \begin{enumerate}
        \item $y=x(x-2)(x+3)$ (3 marks)
        \item $y=(x+1)^2(2-x)$ (3 marks)
    \end{enumerate}
\end{enumerate}


\end{document}